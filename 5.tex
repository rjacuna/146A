\documentclass{article}
\usepackage{fontspec}
\usepackage{xcolor}
\usepackage{sagetex}

\usepackage{euler}
\usepackage{amsmath}
\usepackage{unicode-math}


\usepackage[makeroom]{cancel}
\usepackage{ulem}

\setlength\parindent{0em}
\setlength\parskip{0.618em}
\usepackage[a4paper,lmargin=1in,rmargin=1in,tmargin=1in,bmargin=1in]{geometry}

\setmainfont[Mapping=tex-text]{Helvetica Neue LT Std 45 Light}

\newcommand\N{\mathbb{N}}
\newcommand\Z{\mathbb{Z}}
\newcommand\R{\mathbb{R}}
\newcommand\A{\mathbb{A}}

\begin{document}

\begin{center}
  146A---5

  Ricardo J. Acuna

  (862079740)
\end{center}\vspace{1.618em}

\[\int_0^\infty x^2 +y^2 dxdy\]

Using Sage\TeX, one can use Sage to compute things and put them into
your \LaTeX{} document. For example, there are
$\sage{number_of_partitions(5)}$ integer partitions of $5$.
You don't need to compute the number yourself, or even cut and paste
it from somewhere.

Here's some Sage code:

\begin{sageblock}
    f(x) = exp(x) * sin(2*x)
\end{sageblock}

The second derivative of $f$ is

\[
  \frac{\mathrm{d}^{2}}{\mathrm{d}x^{2}} \sage{f(x)} =
  \sage{diff(f, x, 2)(x)}.
\]

Here's a plot of $f$ from $-1$ to $1$:

\sageplot{plot(f, -1, 1)}Using Sage\TeX, one can use Sage to compute things and put them into
your \LaTeX{} document.

\end{document}
